\section{SLAM}
\ac{slam} ist ein bekanntes Problem in der Robotertechnik. 
Im Folgenden wird das Problem selbst erläutert und näher auf die Umsetzung von \ac{slam} im Rahmen dieser Studienarbeit eingegangen.

\subsection{Was ist SLAM?}
Bei dem \ac{slam}-Problem handelt es sich um das Problem, eine Karte einer unbekannten Umgebung zu erstellen.
Gleichzeitig soll die aktuelle Position des Roboters in dieser Karte ermittelt und dargestellt werden.
Hierzu wird ausschließlich die Sensorik des Roboters genutzt.

\subsection{Mapping}
Damit das Fahrzeug Hindernisse umfahren kann, muss es die Position der Hindernisse reltiv zu seiner eigenen kennen.
Hierzu können aktuelle Daten der Sensorik verwendet werden.
Durch das ausschließliche Nutzen der Daten in Echtzeit, wird der Bereich, in dem ein Pfad berechnet werden kann, jedoch stark eingeschränkt.
Eine sinnvolle Lösung zur Vergrößerung des Radius der bekannten Umgebung, ist die Konstruktion einer Karte.
Dies ermöglicht das Speichern von Informationen aus vorherigen Scans.

Für den Aufbau einer solchen Karte gibt es verschiedene Ansätze, welche im Folgenden näher beleuchtet werden.

\paragraph{Matrix} \mbox{}\\
Da es sich, aufgrund des verwendeten \ac{lidar}-Sensors, in dieser Arbeit um eine 2D-Karte handelt, 
ist der simpelste Weg eine Karte umzusetzten eine einfache Matrix.
In dieser steht für jede Koordinate der Karte eine 1, -1 oder 0.

Die Matrix wird mit 0 gefüllt.
Erkennt die Sensorik ein Hinderniss an einer Koordinate (X,Y), wird der Wert an der Stelle in der Matrix auf 1 gesetzt.
Die Werte zwischen dem Hinderniss und der aktuellen Position des Sensors können auf -1 gesetzt werden,
da keine Hindernisse in diesem Bereich erkannt wurden.
Somit kann, solange die Position des Fahrzeugs bekannt ist, die Karte Scan für Scan gefüllt werden.

\paragraph{Grid Map} \mbox{}\\
Im Falle dieser Arbeit, werden sämtliche Distanzen in Millimetern angegeben.
Jedes Feld der Matrix entspricht also einem 1x1 Millimeter großem Quadrat der Umgebung.
Jedoch hat die Sensorik eine begrenzte Auflösung.
Die gescannten Punkte können also, vor allem bei größeren Entfernungen, viele Millimeter oder Zentimeter voneinander entfernt sein.
In Folge dessen kann es dazu führen, dass eine Wand als viele einzelne Punkte erkannt wird.

Zur Lösung dieses Problems kann ein Grid mit niedrigerer Auflösung verwendet werden.
Die Berechnungen finden somit weiter in Millimetern statt, die Karte selbst wird jedoch als Grid in einer niedrigeren Auflösung gespeichert.
Hat das Grid zum Beispiel eine Auflösung von 1x1 Zentimeter, werden gescannte Punkte auf Zentimeter gerundet bevor die Information in der Karte gespeichert wird.

\paragraph{Wahrscheinlichkeiten}
Eine weitere sinnvolle Ergänzung ist die Verwendung von Wahrscheinlichkeiten.
Anstelle von 1, -1 und 0 werden auch sämtliche Werte dazwischen genutzt.

Wird ein Hinderniss erkannt, wird der entsprechende Wert im Grid nicht auf 1 gesetzt.
Stattdessen wird der Wert mit jedem mal wo ein Hinderniss in dem Grid erkannt wird, leicht erhöht.
Gleichzietig wird der Wert leicht verringert, wenn ein Punkt in dem Grid als frei erkannt wird.
Zusätzlich werden Schwellenwerte definiert, die festlegen, ab wann ein Feld des Grids als belegt oder frei gilt.

Das Ergebnis ist eine Occupancy Grid Map, welche eine geringere Auflösung als die gesamte Karte hat.
In ihr werden Informationen über den Status des Teils der Karte gepseichert, welche von dem Feld der Grid Map repräsentiert wird.
\ref{fig:gridMap}

\begin{figure}[H]
    \centering
    \subfloat[a][Simulierte Umgebung]{
        \includegraphics[width=10cm]{graphics/umgebung.png}
    } \\
    \subfloat[b][Occupancy Grid Map]{
        \includegraphics[width=10cm]{graphics/map.png}
    }
    \caption{Umgebung auf Occupancy Grid Map abegbildet}
    \label{fig:gridMap}
\end{figure}

Wie in der Abbildung \ref{fig:gridMap} zu sehen ist, wird die Umgebung (a) gescannt und in eine Grid Map (b) abgebildet.
Die Map selbst ist mit 2000mm x 2000mm doppelt so groß wie die Umgebung, das Grid hat jedoch nur eine Auflösung von 500px x 500px.
Das bedeutet, dass jeder Pixel des Grid ein 4mm x 4mm Quadrat der Map darstellt.

\subsection{Lokalisierung}
Das selbstfahrende Fahrzeug soll in der Lage sein ein vorgegebenes Ziel zu erreichen. 
Daher ist neben dem Mapping auch die Lokalisierung des Fahrzeuges eine zentrale Aufgabe. 
Denn ohne das Wissen über die aktuelle Position auf der Karte kann kein Weg zum Ziel berechnet werden
und es kann auch nicht bestimmt werden ob das Ziel erreicht ist.

In Folgenden werden verschiedene Ansätze zur Lösung des Lokalisierungsproblems mit den Vor- und Nachteilen beschrieben.

\subsubsection{Datengenerierung}
In diesem Abschnitt werden verschiedene Möglichkeiten zur generierung von Daten, welche zur Lokalisierung eines Fahrzeugs genutzt werden können, näher betrachtet.

\paragraph{Datenquellen zur Erzeugung globaler Bewegungsdaten} \mbox{}\\
Globale Verfahren zur Datenerzeugung für Lokalisierungsalgorithmen basieren darauf dass die Umgebung dafür prepariert ist.
Das Fahrzeug kommuniziert mit Sendern die in der Umgebung verfügbar sind. 
Mit Hilfe dieser Sender wird ein globales Netz erstellt, dass dem Fahrzeug die Berechnung der absoluten Position im 
globalen Koordinatensystem ermöglicht.

\begin{enumerate}[leftmargin=*]
    \item \textbf{\acf{gps}} \\
    Für die Ortung eines Fahrzeugs kommt in der Praxis das \ac{gps} zum Einsatz. 
    \ac{gps} arbeitet mit Satelliten, die die Position des Benutters, in diesem Fall des Fahrzeugs, bestimmen und übermittlen \cite{ashby2003relativity}. 
    Die Genauigkeit des \ac{gps} beträgt ca. 5-10 cm \cite{ashby2003relativity}.

    Der Vorteil der Nutzung eines \ac{gps}-Sensors ist eine globale Verfügbarkeit.

    Im Einsatz für Fahrzeuge auf der Straße ist diese Genauigkeit ausreichend, 
    da die lokalisierten Objekte deutlich größer sind und dadurch trotz der Toleranzen der richtige Ort gefunden werden kann.
    Relativ zur Fahrzeuggröße sind 5-10 cm bei einem kleinen Modellfahrzeug eine deutliche Abweeichung, die abhängig von der Umgebung des Fahrzeugs ernsthafte Konsequenzen haben kann.

    Eine weitere Problematik die die Verwendung von \ac{gps}-Daten mit sich bringt ist die Abhängigkeit von der Signalstärke und -verfügbarkeit. 
    Ist das Siganl schewach, kann die Abweeichung noch größer werden. 
    Ist kein Siganl verfügbar, ist gar keine Ortung möglich.
    
    \item \textbf{Eigenes \ac{gps}} \\
    Um das Problem der Siganlverfügbarkeit zu lösen, könnte man auf die Idee kommen ein eigenes \acf{gps} aufzubauen, dass kleine Sender statt Satelliten verwendet.
    Diese Sender werden in der Umgebung platziert. 
    Auf dem Fahrzeug ist ein Empfänger moniteirt, der die Entfernungen zu den Sendern erfasst.
    Mit dieser Technologie kann dann über Triangulation die Position des Fahrzeugs bestimmt werden. 
    Dadurch wäre je nach Qualität von Sender und Empfänger eine höhere Präzision als 5-10 cm möglich. 
    
    Damit wären also beide Probleme von \ac{gps} in diesem Kontext gelöst. 
    Aber es gibt auch einen deutlichen Nachteil. 
    Denn vor der Verwendung des Fahrzeugs muss die Umgebung zunächst mit den Sendern präpariert werden. 
    Ein Einsatz in unbekannten Gebieten ist dadurch nicht möglich. 
    Je nach Einsatzzweck des Fahrzeuges ist das ein großes Problem.
\end{enumerate}

\paragraph{Quellen für die Erzeugung relativer Bewegungsdaten} \mbox{}\\
Globale Ortungsverfahren haben den Nachteil abhängig von den Gegebenheiten der Umgebung zu sein.
Ist in der Umgebung keine Kommunikation mit den Sendern möglich, so ist keine Lokalisierung möglich.
Dabei ist es nicht von Bedeutung ob das Signal von Satelliten oder von selbst angebrachten Sendern in der Umgebung stammt.
Aus diesem Grund gibt es auch relative Verfahren, die die Positionsänderung anhand von Differenzen in den gesammelten Daten von verbauten Sensoren berechnen.

\begin{enumerate}[leftmargin=*]
    \item \textbf{LiDAR-Sensor} \\
    Bei einem LiDAR-Sensor wird die Umgebung mit Hilfe von Laserstrahlen erfasst.
    Dabei werden Werte generiert, die die Distanz und Richtung des Objektes beinhalten. 
    LiDAR-Sensoren bieten den Vorteil, dass sie unabhängig von den Lichtverhältnissen der Umgebung sind \cite{lidar}. 
    Bei LiDAR-Sensoren muss zwischen 2D- und 3D-Sensoren differenziert werden. 
    3D-Sensoren erfassen eine deutlich größere Datenmenge, wodurch die Verlässlichkeit der Daten erhöht wird.
    Dabei spielt es keine Rolle, ob der Sensor im Innen- oder Außenbereich zum Einsatz kommt \cite{lidar}. 

    Ein weiterer Vorteil von LiDAR-Sensoren ist die Datenrepräsentation.
    Die Repräsentation mit Abstand und Winkel ermnöglicht eine Verarbeitung ohne aufwendige Vorbereitung und Anpassung der Daten.
    
    Die Präzision von LiDAR-Sensoren kann aber zum Beispiel unter Wettereinflüssen leiden. 
    Zum Beispiel können Wasserteilchen in der Luft die Reflektion der Laserstrahlen so verändern, dass die empfangenen Werte des Sensors nicht mit der Realität übereinstimmen.
    Auch die eingeschränkte Reichweite kann abhängig von der Umgebung ein Problem darstellen.

    Bei 2D-Sensoren kann auch die feste Höhe ein Problem sein.
    Ist ein Hinderniss unter- oder überhalb des Sensors, aber auf Höhe anderer Farzeugteile, 
    werden diese nicht erkannt und eine Kollision mit solchen Objekten kann nicht verhindert werden. 

    \item \textbf{Kamera} \\
    Kameras haben den Vorteil, dass alle Elemente, unabhängig von der Höhe, und auch in größeren Distanzen erkannt werden können. 

    Der Nachteil von Kameras ist die Abhängigkeit von der Beleuchtung der Umgebung, da diese maßgeblich den erkennabren Detailgerad beeinflusst.
    Auch Wetterfaktoren wie Nebel oder Niederschlag können die Qualität der Daten negativ beeinflussen.

    Auch die Verarbeitung der Daten ist ein Nachteil. 
    Um Kameradaten automatisiert auszuwerten müssen zunächst verschiedene Algorithmen zur Vorbereitung ausgeführt werden. 
    Die Vorbereitung benötigt Zeit, die bei anderen Verfahren bereits zur Berechnng der Posiiton genutzt werdeen kann.
    Außerdem sind die Vorbereitungen und die Auswertung der Biler rechenintnsiv, 
    wodurch diese als primäre Datenquelle zur Lokalisierung eher für Fahrzeuge mit hoher Rechenleistung geeignet sind.

    Ein weiterer Nachteil von Kameras ist die eingeschränkte Sichtweite. 
    Das Bild kann nur einen gewissen Teil der Umgebung aufnehmen und bietet nur durch die Kombination mehrerer Kameras eine vollständige Wahrnehmung der Umgebung.


    \item \textbf{Ultraschall} \\
    Ultraschallsensoren erfassen die Umgebung mit Hilfe von Schallwellen und deren Reflektionen. 
    Diese Methode ist sehr einfach in der implementierung und sehr sparsam im Energieverbrauch. 
    Daher ist ein Einsatz auch in Fahrzeugen mit geringer Batterikapazität möglich.
    Das hat aber auch Nachteile. 
    Die Reichweite von Ultraschallsensoren ist geringer als die anderer Sensoren. 
    Außerdem ist auch die Auflösung der erfassten Daten geringer als die anderer Sensoren.

    Daher ist der Einsatz als primäre Datenquelle für die Lokalisierung des Fahrzeuges nur bedingt geeignet.
    Die Stärken von Ultraschall liegen eher im Nahbreich.
    Ein Einsatz dieser Technologie wäre also als Zusatz zu einer anderen Quelle denkbar. 
    Das Ziel wäre dann durch die Ultraschall-Daten die Präzision der berechneten Position durch die primäre Datenquelle zu erhöhen. 
\end{enumerate}

\subsubsection{Point Cloud Registration}
\label{pcl}
Point Cloud Registration beschreibt ein Problem zur Schätzung der Transformation zwischen mehreren Punktewolken. \ref{fig:registrationExample}

Mit Hilfe von Punktewolken können beliebig große Mengen an Punkten dargestellt werden.
Die Wolke beinhaltet Informationen zu jedem der Punkte.
Hierzu gehören zumindest die Koordinaten.
Eine solche Wolke kann jedoch auch weitere Informationen wie Farbe oder Krümmung beinhalten. 
Die Gesamtheit der Punkte inerhalb einer solchen Wolke beschreibt die Form und Oberfläche eines Objektes
\cite[chapters 2.2]{registration2021}

Registration selbst lässt sich in verschiedene Unterkategorien einteilen.
Beschränkt sich die Transformation auf Rotation und Translation, spricht man von einer steifen Transformation bzw. Registration.
Des Weiteren wird anhand der Quelle und Anzahl der Datensätze unterschieden.

Die Punktewolken, welche Teil dieser Arbeit sind, werden mit Hilfe eines einzelnen, zwei-dimensionalen \ac{lidar}-Sensor erstellt.
Sie enthalten ausschließlich Informationen über die X und Y Koordinaten der einzelnen Punkte.

Somit spricht man bei der, im Kontext dieser Arbeit durchgeführten Point Cloud Registration, von einer paarweisen und steifen 2D Registration.
Diese Art der Registration ist eine der simpelste, da lediglich zwei Punktewolken des selben Sensors miteinander verglichen werden 
und ausschließlich die Translation auf den zwei Achsen sowie die Rotation berechnet werden muss.

\begin{figure}[H]
    \centering
    \subfloat[a][Source und Target Point Cloud]{
        \includegraphics[width=12cm]{graphics/registration_source_target.png}
    } \\
    \subfloat[b][Transformierte Source Point Cloud]{
        \includegraphics[width=8cm]{graphics/registration_alligned.png}
    }
    \caption{Beispiel einer starren 2D Point Cloud Registration}
    \label{fig:registrationExample}
\end{figure}

\paragraph{Ablauf} \mbox{}\\
Der Ablauf einer solche Registration beinhaltet, wie in \ref{fig:registrationAblauf} beschrieben, typischerweise sechs Schritte.
\begin{enumerate}[leftmargin=*]
    \item Datenerfassung
    \item Schätzung der Keypoints
    \item Schätzung der Feature-Deskriptoren
    \item Schätzung der Korrespondenzen (Matching)
    \item Ablehnung von Korrespondenzen
    \item Schätzung der Transformation
\end{enumerate}

\begin{figure}[H]
    \centering
    \includegraphics[width=12cm]{graphics/registration_ablauf.png}
    \caption{Typischer Ablauf einer paarweisen Registration. Quelle: \cite{pcl2023}}
    \label{fig:registrationAblauf}
\end{figure}

Datenerfassung: \\
Die Datenerfassung kann auf unterschiedlichste Arten erfolgen.
Hierbei werden zwei Sets an Daten, gesammelt.
Die Daten werden verarbeitet und in einer Punktewolke gespeichert.
Durch ein einheitliches Vormat wird sichergestellt, dass die Punkte korrekt weiterverarbeitet werden.
\newline

Schätzung der Keypoints: \\
Die Schätzung von Keypoints ist von enormer Wichtigkeit.
Sie dient der Verringerung notwendiger Rechenleistung.

Möchte man zwei Scans mit jeweils 100 Tausend Punkten vergleichen, gibt es 10 Miliarden mögliche Korrespondenzen.
Um die Zahl der Korrespondenzen zu verringern, werden Keypoints in des Scans gesucht.

Ein Keypoint beschreibt einen Punkt, welcher spezielle Eigenschaften innerhalb der Szene haben.
Eine Beispiel hierfür wäre eine Ecke.

Die Keypoints werden im weiteren Verlauf für die Berechnungen genutzt, wodurch sich die Anzahl an Punkten drastisch senkt.
Im Optimalfall ist das Ergebnis genau das selbe, benötigt aber deutlich weniger Rechenleistung und somit Zeit.
\newline

Schätzung der Feature-Deskriptoren: \\
Je nach Anwendungszweck sind Koordinaten nicht ausreichend um einen Punkt zu beschreiben.
Feature-Deskriptoren oder Point-Feature Repräsentationen sind eine Form der erweiterten Beschreibung eines Punktes.

Durch miteinbeziehen der umliegenden Punkte, können Informationen über die Form und Beschaffenheit der Fläche gesammelt werden.
Diese können widerum in den Feature-Deskriptoren gespeichert werden.
Die simpelste Form eines solchen Feature-Deskriptor wäre die Normale der Fläche unter dem Punkt.
\newline

Schätzung der Korrespondenzen: \\
Die zwei vorhandenen Sets von Feature-Deskriptoren, welche aus den Keypoints berechnet wurden, 
können nun verwendet werden um Korrespondenzen zu schätzen.
Bei kleineren Datensets kann es auch Sinn machen, die Keypoint- und Feature-Deskriptor-Schätzung auszulassen
und die Korrespondenzen nur mittels Koordinaten der Punkte zu schätzen.

Eine Korrespondenz beschreibt zwei Punkte oder Feature-Deskriptoren aus verschiedenen Datensätzen, welche den gleichen Punkt im Raum repräsentieren.
\newline

Ablehnung von Korrespondenzen: \\
Nachdem die Korrespondenzen geschätzt wurden, müssen schlechte Korrespondenzen verworfen werden.

Hierzu gibt es diverse Algorithmen, worunter der RANSAC Algorithmus am weitesten verbreitet ist.
Eine weitere Möglichkeit die Anzahl an Korrespondenzen zu senken, 
ist das Filtern von Korrespondenzen die zwar den gleichen Source-Punkt,
aber unterschiedlichen Punkten im Ziel-Datensatz korrespondieren.
Hierbei kann die Korrespondenz mit der kleinsten Distanz gewählt werden.
Die anderen Korrespondenzen werden verworfen.
\newline

Schätzung der Transformation: \\
In einem Finalen Schritt wird die Transformationsmatrix geschätzt.
Diese Matrix beschreibt die Translation und Rotation, 
welche notwendig ist um die Punktewolke A zur Punktewolke B zu transformieren.

Die Schätzung der Transformation passiert mittels, auf den Korrespondenzen basierenden, Metriken.
Ein Beispiel für eine solche Metrik ist die mittlere qudratische Abweichung der Korrespondenzen.
Die Punktewolke wird transformiert und die Metriken ausgewertet.
Dieser Vorgang wird solange wiederholt bis ein Abbruch-Kriterium erfüllt ist.
Ein solches Abbruch-Kriterium ist z.B. das Unterschreiten eines Grenzwertes für die mittlere Qudratische Abweichung.
Auch eine Überschreitung einer bestimmten Anzahl an Iterationen kann zum Abbruch führen.

\paragraph{Point Cloud Registration im Rahmen dieser Arbeit} \mbox{}\\
Die Datensätze dieser Arbeit sind mit unter 1000 Punkten recht klein.
Somit ist eine Schätzung von Keypoints wenig Sinnvoll und teilweise auch nicht umsetzbar.

In der Theorie verringert das Berechnen von Feature-Deskriptoren die Laufzeit und verbessert das Ergebnis.
Selbst durchgeführten Tests ergaben jedoch, dass die Laufzeit sich verschlechterte 
und das Ergbenis auch ohne die Nutzung von Feature-Deskriptoren ausreichend genau ist.
Das lässt sich durch die geringe Anzahl an Punkten innerhalb unserer Datensätze und der Nutzung von lediglich zwei Dimensionen erklären.
Eine solche Berechnung ist also, im Rahmen der Arbeit, ebenfalls wenig Sinnvoll.

Für die Schätzung der Korrespondenzen sowie der Ablehnung schlechter Korrespondenzen 
und der Schätzung der Transformationsmatrix wird eine Implementierung des \acf{icp}-Algorithmus der \acf{pcl} verwendet.
Die \ac{pcl} ist eine C++ Bibliothek, die ein effizientes Arbeiten mit Punktewolken ermöglicht
und diverse Algorithmen für unterschiedlichste Operationen mitbringt.
Genauere Infos zur Implementierung des \ac{icp}-Algorithmus Kapitel \ref{slamImplementierung}.

\newpage
