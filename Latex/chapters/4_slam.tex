\section{SLAM}
\ac{slam} ist ein bekanntes Problem in der Robotertechnik. 
Im Folgenden wird das Problem selbst erläutert und näher auf die Umsetzung von \ac{slam} im Rahmen dieser Studienarbeit eingegangen.

\subsection{Was ist SLAM?}
Bei dem \ac{slam}-Problem handelt es sich um das Problem, eine Karte einer unbekannten Umgebung zu erstellen.
Gleichzeitig soll die aktuelle Position des Roboters in dieser Karte ermittelt und dargestellt werden.
Hierzu wird ausschließlich die Sensorik des Roboters genutzt.

\subsubsection{Mathematische Betrachtung}


\subsection{Mapping}
%% Occupancy Grid

\subsection{Lokalisierung}
Das selbstfahrende Fahrzeug soll in der Lage sein ein vorgegebenes Ziel zu erreichen. 
Um diese Aufgabe zu meistern ist die Lokalisierung des Fahrzeuges eine zentrale Aufgabe. 
Denn ohne das Wissen über den aktuellen Standort kann kein Weg zum Ziel berechnet werden 
und es kann auch nicht bestimmt werdeen ob das Ziel erreicht ist.

In diesem Abschnitt werden verschiedene Ansätze zur Lösung des Lokalisierungsproblems mit den Vor- und Nachteilen beschrieben.

\subsubsection{Globale Ortungsverfahren}

\begin{enumerate}
    \item \textbf{\acf{gps}}\\
    Für die Ortung eines Fahrzeugs kommt in der Praxis das \ac{gps} zum Einsatz. 
    \ac{gps} arbeitet mit Satelliten, die die Position des Benutters, in diesem Fall des Fahrzeugs, bestimmen
    und übermittlen \cite{ashby2003relativity}. Die Genauigkeit des \ac{gps} beträgt ca. 5-10 cm \cite{ashby2003relativity}.

    Der Vorteil der Nutzung eines \ac{gps}-Sensors ist eine globale Verfügbarkeit.

    Im Einsatz für Fahrzeuge auf der Straße ist diese Genauigkeit ausreichend, da die lokalisierten Objekte deutlich größer sind und dadurch
    trotz der Toleranzen der richtige Ort gefunden werden kann.
    Relativ zur Fahrzeuggröße sind 5-10 cm bei einem kleinen Modellfahrzeug eine deutliche Abweeichung, die abhängig von der Umgebung des Fahrzeugs ernsthafte Konsequenzen haben kann.

    Eine weitere Problematik die die Verwendung von \ac{gps}-Daten mit sich bringt ist die Abhängigkeit von der Signalstärke und -verfügbarkeit. 
    Ist das Siganl schewach, kann die Abweeichung noch größer werden. Ist kein Siganl
    verfügbar, ist gar keine Ortung möglich.
    \item \textbf{Eigenes \ac{gps}}\\
    Um das Problem der Siganlverfügbarkeit zu lösen, könnte man auf die Idee kommen ein eigenes \acf{gps} aufzubauen, dass kleine Sender statt Satelliten verwendet.
    Diese Sender werden in der Umgebung platziert. 
    Auf dem Fahrzeug ist ein Empfänger moniteirt, der die Entfernungen zu den Sendern erfasst.
    Mit dieser Technologie kann dann über Triangulation die Position des Fahrzeugs bestimmt werden. Dadurch wäre je nach Qualität von Sender und Empfänger eine höhere Präzision als 
    5-10 cm möglich. 
    
    Damit wären also beide Probleme von \ac{gps} in diesem Kontext gelöst. Aber es gibt auch einen deutlichen Nachteil. Denn vor der Verwendung des Fahrzeugs muss die Umgebung zunächst
    mit den Sendern präpariert werden. Ein Einsatz in unbekannten Gebieten ist dadurch nicht möglich. Das ist je nach Einsatzzweck des Fahrzeuges ein größeres oder kleiners Problem.
\end{enumerate}

\subsubsection{Relative Ortungsverfahren}
Globale Ortungsverfahren haben den Nachteil abhängig von den Gegebenheiten der Umgebung zu sein.
Ist in der Umgebung kein \ac{gps}-Signal verfügbar, so ist keine Lokalisierung möglich.
Dabei ist es nicht von Bedeutung ob das Signal von Satelliten oder von selbst angebrachten Sendern in der Umgebung stammt.
Aus diesem Grund gibt es auch relative Verfahren, die die Positionsänderung anhand von Differenzen in den gesammelten Daten 
von verbauten Sensoren berechnen.

\begin{enumerate}
    \item \textbf{LiDAR-Sensor}\\
    Bei einem LiDAR-Sensor wird die Umgebung mit Hilfe von Laserstrahlen erfasst Dabei werden Werte generiert, die die Distanz und Richtung des Objektes beinhalten. 
    LiDAR-Sensoren bieten den Vorteil, dass sie unabhängig von den Lichtverhältnissen der Umgebung sind \cite{lidar}. 
    Bei LiDAR-Sensoren muss zwischen 2D- und 3D-Sensoren differenziert werden. 3D-Sensoren erfassen eine deutlich größere Datenmenge, 
    wodurch die Verlässlichkeit der Daten erhöht wird. Dabei spielt es keine Rolle, ob der Sensor im Innen- oder Außenbereich zum Einsatz kommt \cite{lidar}. 

    Ein weiterer Vorteil von LiDAR-Sensoren ist die Datenrepräsentation.
    Die Repräsentation mit Abstand und Winkel ermnöglicht eine Verarbeitung ohne aufwendige Vorbereitung und Anpassung der Daten.
    
    Die Präzision von LiDAR-Semsoren kann aber zum Beispiel unter Wettereinflüssen leiden. 
    Zum Beispiel können Wasserteilchen in der Luft die Reflektion der Laserstrahlen so verändern, dass die empfangenen Werte des Sensors nicht mit der Realität übereinstimmen.
    Auch die eingeschränkte Reichweite kann abhängig von der Umgebung ein Problem darstellen.

    Bei 2D-Sensoren kann aucb die feste Höhe ein Problem sein.
    Ist ein Hinderniss unter- oder überhalb des Sensors, aber auf Höhe anderer Farzeugteile, werdeen diese nicht erkannt und eine Kollision mit 
    solchen Objekten kann nicht verhindert werden. 

    \item \textbf{Kamera}\\
    Kameras haben den Vorteil, dass alle Elemente, unabhängig von der Höhe, und auch in größeren Distanzen erkannt werden können. 

    Der Nachteil von Kameras ist die Abhängigkeit von der Beleuchtung der Umgebung, da diese maßgeblich den erkennabren Detailgerad beeinflusst.
    Auch Wetterfaktoren wie Nebel oder Niederschlag können die Qualität der Daten negativ beeinflussen.

    Auch die Verarbeitung der Daten ist ein Nachteil. 
    Um Kameradaten automatisiert auszuwerten müssen zunächst verschiedene Algorithmen zur Vorbereitung ausgeführt werden. 
    Die Vorbereitung benötigt Zeit, die bei anderen Verfahren bereits zur Berechnng der Posiiton genutzt werdeen kann.
    Außerdem sind die Vorbereitungen und die Auswertung der Biler rechenintnsiv, wodurch diese als primäre Datenquelle zur Lokalisierung 
    eher für Fahrzeuge mit hoher Rechenleistung geeignet sind.

    Ein weiterer Nachteil von Kameras ist die eingeschränkte Sichtweite. 
    Das Bild kann nur einen gewissen Teil der Umgebung aufnehmen und bietet nur durch die Kombination mehrerer Kameras eine vollständige Wahrnehmung der Umgebung.


    \item \textbf{Ultraschall}\\
    Ultraschallsensoren erfcassen die Umgebung mit Hilfe von Schallwellen und deren Reflektionen. 
    Diese Methode ist sehr einfach in der implementierung und sehr sparsam im Energieverbrauch. 
    Daher ist ein Einsatz auch in Fahrzeugen mit geringer Batterikapazität möglich.
    Das hat aber auch Nachteile. 
    Die Reichweite von Ultraschallsensoren ist geringer als die anderer Sensoren. 
    Außerdem ist auch die Auflösung der erfassten Daten geringer als die anderer Sensoren.

    Daher ist der Einsatz als primäre Datenquelle für die Lokalisierung des Fahrzeuges nur bedingt geeignet.
    Die Stärken von Ultraschall liegen eher im Nahbreich.
    Ein Einsatz dieser Technologie wäre also als Zusatz zu einer anderen Quelle denkbar. 
    Das Ziel wäre dann durch die Ultraschall-Daten die Präzision der berechneten Position durch diue primäre Datenquelle zu erhöhen. 
\end{enumerate}


\paragraph{ICP}

\newpage
