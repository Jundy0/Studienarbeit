\section{Problemstellung, Ziel und Umsetzung}
In diesem Abschnitt wird auf die Problemstellung, das generelle Ziel und die geplante Umsetzung der Arbeit eingegangen.
Des Weiteren wird erläutert, weshalb das Ziel der Arbeit wichtig ist, wie die Arbeit aufgebaut ist und welche Probleme und Schwierigkeiten durch bereits getätigte Versuche einer Umsetzung des Arbeits-Ziels bereits bekannt sind.

\subsection{Problemstellung}
Bisherige Versuche, ein selbstfahrendes Auto, im Rahmen einer Studienarbeit zu entwerfen und einen entsprechenden Algorithmus zu programmieren, sind gescheitert. 
Die beiden Hauptprobleme der bisherigen Arbeiten, war zum einen der Bau eines geeigneten Fahrzeugs und zum anderen die Entwicklung einer Software, welche es dem Auto ermöglicht, 
ohne manuelle Steuerung, Hindernisse zu erkennen und um diese herum zu navigieren.
Da es sich bei den Studenten der bisherigen Gruppen ausschließlich um Studenten mit einem Schwerpunkt in Elektrotechnik handelte, war die Entwicklung und Implementierung der Software zum autonomen Fahren die größere Herausforderung.

Da keine der bisherigen Gruppen, das Problem der Software lösen konnte, wurde sich dazu entschieden, die Aufgaben aufzuteilen. 
Die Aufgabe, der Erstellung eines funktionsfähigen Modell-Autos und einer Schnittstelle, zur Steuerung des Autos, wurde einer Gruppe von Studenten mit einem Schwerpunkt in Elektrotechnik zugeteilt. 
Somit ist das Hauptproblem dieser Arbeit, die Entwicklung und Implementierung eines Ausweichalgorithmus, 
welcher das Auto, über die, von der anderen Gruppe zur Verfügung gestellten Schnittstelle, steuern soll und so eine autonome Hindernis-Detektierung und Vermeidung ermöglicht.

Da die Aufgabe auf mehrere Gruppen aufgeteilt wurde, ist ein weiteres Problem die Kommunikation zwischen den Gruppen. 
Um einen reibungslosen und effizienten Ablauf gewährleisten zu können, sollte diese möglichst Umfangreich sein.

Durch die Aufteilung auf verschiedene Gruppen entsteht zusätzlich das Problem, dass die Hardware nur eingeschränkt verfügbar ist. 
Da die Gruppe, welche den Hardware-Teil der Aufgabe übernimmt, diese erst bauen und anschließend auch testen muss, ist die Hardware, vor allem zu Beginn der Arbeit, kaum verfügbar.
Daher ist zur Entwicklung der Software, eine Abstrahierung der Hardware notwendig. Konkret bedeutet das, dass die Schnitt\-stelle zur Steuerung, sowie die Daten der Sensoren simuliert werden müssen, 
um ein Testen des Algorithmus auch ohne Verfügbarkeit der Hardware zu ermöglichen.

\subsection{Ziele der Arbeit}
In diesem Abschnitt werden die Ziele der Arbeit und deren Metriken beschrie\-ben.

Das Ziel auf das hingearbeitet wird, ist die Implementierung eines Algorithmus für ein Modell-Fahrzeug. 
Durch den Algorithmus soll es möglich sein, das Fahrzeug autonom durch die Umgebung zu einem Zielpunkt zu steuern.

Da das Auto selbst parallel von einer anderen Gruppe gebaut wird, steht es erst zu einem späteren Zeitpunkt zur Verfügung. 
Daher ist die Entwicklung einer Simulation ein erster Zwischenschritt. 

Um den Fortschritt des Projekts und die Ergebnisse messen zu können, werden Metriken für verschiedene Bereiche definiert.
Die Erfüllung von Zielmetriken ist teilweise abhängig von der Wahl der Hard- und Softwarekomponenten.
Aus diesem Grund werden an dieser Stelle zunächst die Metriken definiert, die unabhängig von externen Faktoren erfüllt werden können. 

\begin{enumerate}[leftmargin=*]
    \item \textbf{Simulation einer Umgebung}\\
    In der Simulation sollen Hindernisse generiert werden können, die von einem simulierten Sensor erfasst werden können.

    \textbf{Metrik:} 
    \begin{itemize}
        \item Kann eine Umgebung generiert werden? $\to$ Ja oder Nein
    \end{itemize}
    
    \item \textbf{Erkennung von Hindernissen}\\
    Die Software muss in der Lage sein alle Hindernisse erkennen die den Spezifikationen, 
    die für die Anpassung an die technischen Gegebenheiten getroffen werden, 
    entsprechen.

    \textbf{Metrik:} 
    \begin{itemize}
        \item Werden alle Hindernisse erkannt? $\to$ Ja oder Nein
    \end{itemize}
    
    \item \textbf{Festlegen eines Zielpunktes}\\
    Um einen Ausweichalgorithmus zu testen, ist die Festlegung eines Ziel\-punktes unumgänglich. 
    Denn durch den Zielpunkt ist klar definiert, wo das Auto ankommen soll und welche Hindernisse auf dem Weg zwischen Fahrzeug und Ziel liegen.

    \textbf{Metrik:} 
    \begin{itemize}
        \item Kann ein Zielpunkt definiert werden? $\to$ Ja oder Nein
    \end{itemize}
    
    \item \textbf{Simulation eines Umgebungssensors}\\
    Um die Hindernisse in der Umgebung zu erkennen, soll ein Sensor simuliert werden, der die Hindernisse erfasst und für die weitere Verarbeitung verfügbar macht.

    \textbf{Metrik:}
    \begin{itemize}
        \item Kann ein Sensor simuliert werden? $\to$ Ja oder Nein
    \end{itemize}
    
    \item \textbf{Lokalisierung des Fahrzeugs}\\
    Um eine Route für das Fahrzeug berechnen zu können, muss eine Algorithmik zur Lokalisierung des Fahrzeugs implementiert werden.

    \textbf{Metrik:} 
    \begin{itemize}
        \item Für die Lokalisierung bezieht sich die Metrik auf Präzision und Ge\-schwindigkeit. 
        Beide Elemente sind abhängig von den gewählten Hard und Softwarekomponenten. 
    \end{itemize}

    \item \textbf{Berechnung der Route}\\
    Damit das Fahrzeug den festgelegten Zielpunkt erreichen kann, muss eine Route berechnet werden, die um alle erkannten Hindernisse herum führt.
    
    \textbf{Metriken:}
    \begin{itemize}
        \item Führt die Route vom Fahrzeug zum Zielpunkt $\to$ Ja oder Nein
        \item Werden alle, in der Karte des Fahrzeuges bekannten, Hindernisse umfahren $\to$ Ja oder Nein
        \item Kann die gesamte Route mit dem Fahrzeug abgefahren werden? $\to$ Ja oder Nein
        \item Wie lange dauert die Berechnung der Route. Berechnungsdauer < 100 ms
    \end{itemize}

    \item \textbf{Steuerung des Fahrzeuges}\\
    Um das Auto entlang der berechneten Route zu bewegen, muss es von der Software angesteuert werden können.

    \textbf{Metrik:}
    \begin{itemize}
        \item Kann das Auto angesteuert werden? $\to$ Ja oder Nein
    \end{itemize}
\end{enumerate}

\subsection{Umsetzung}
Um eine umfangreiche Kommunikation zwischen den Gruppen zu ermögli\-chen, müssen Kommunikationswege so früh wie möglich erstellt werden.
Des Weiteren sollten Termine für regelmäßige Meetings festgelegt werden, um einen konstanten Austausch von Informationen zwischen den Gruppen zu gewähr\-leisten.

Zur Umsetzung der Aufgabe selbst, stehen, neben einem RPLiDAR A1M8-R6 der Firma Slamtec, auch weitere Sensoren, 
wie Ultraschall-, Lenkwinkel- und Geschwindigkeits-Sensoren, sowie ein Raspberry PI 4 zur Verfügung.
Bevor die eigentliche Arbeit an einem Algorithmus beginnen kann, muss die gegebene Hardware getestet werden. 
Zudem ist es, um das weitere Vorgehen planen zu können, notwendig, sich mit der Hardware vertraut zu machen. 
Zu wissen, welche Daten von den Sensoren, wann gesendet werden, ermöglicht es, präziser zu planen, wodurch die Entwicklung des Algorithmus effizienter wird.

Nachdem verstanden wurde, wie die Hardware funktioniert, muss eine Mög\-lichkeit, diese zu simulieren, entwickelt werden.
Hierbei ist es wichtig, die, für den Algorithmus notwendige Hardware, so genau wie möglich zu simulieren. 
Je genauer die Simulation ist, desto unwahrscheinlicher treten Probleme bei der Zusammenführung von Hard- und Software auf.

Die Simulation dient jedoch nur zum Testen des Algorithmus.
Im späteren Betrieb sollen, anstelle der Daten der Simulation, die Daten der vorhandenen Sensorik verwendet werden.
Hierzu ist die Entwicklung eines Interface, durch welches mit der Sensorik kommuniziert werden kann, notwendig.

Nachdem nun realitätsnahe, simulierte Daten, sowie echte Daten, eingelesen werden können, kann die Umsetzung eines \acf{slam}-Algorithmus begonnen werden.
Hierzu muss eine Möglich\-keit entwickelt werden, mit der die vorhandenen Daten zur Erstellung einer Karte genutzt werden können.
Des Weiteren muss das Auto innerhalb der Karte lokalisiert werden können.

Als Nächstes muss der Ausweichalgorithmus entwickelt werden.
Dieser muss in der Lage sein, die vorhandenen Daten zu nutzen und so das Auto um Hindernisse herum zu einem gewünschten Zielort zu navigieren.

Abschließend wird die Hardware und die Software vereint und getestet.

\newpage
