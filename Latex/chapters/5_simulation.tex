\section{Simulation des Ausweichalgorithmus}

In diesem Kapitel wird beschrieben, wwarum eine Simulation hilfreich für die Entwicklung eines 
sicherheitskritischen Algorithmus, wie zum Beispiel ein Ausweichalgorithmus, sein kann.
Ausßerdem wird beschrieben, welche Aspekte der Simulation relevant für den Übertrag der Ergebnisse auf die Realität sind.

 \subsection{Was ist eine Simulation?}
 \label{Simulation}
 Bevor damit begonnen werden kann die verschiedenen Aspekte einer Simulation zu beleuchten, ist zu klären, was eine Simulation ist.
 Nach der Aussage von A. Maria ist eine Simulation eine Ausführung eines Modells eines Systems \cite{maria1997introduction}[p. 1, ch. 2].
Der Begriff des Modells wird ebenfalls in der Arbeit beschrieben. Ein Modell ist eine vereinfachte, funktionierende Repräsentation des Systems, das betrachtet werden soll \cite{maria1997introduction}[p. 1, ch. 1].\\
In der Simulationstechnik gibt es unterschiedliche Arten von Simulation. In diesem Kontext von Bedeutung ist die Unterscheidung zwischen realer Simulation und Computersimulation. Reale Simulationen kommen zum Einsatz, wenn durch einen Fehler keine Gefahr für Personen und Umwelt besteht.
Außerdem kann es sein, dass ein Nachstellen der Umweltbedingungen so komplex ist, dass eine Nachbildung am Computer nicht ausreichend möglich oder zeitlich zu aufwendig ist. Computersimulationen kommen dann zum Einsatz, wenn ein Fehler schädliche Folgen für Perosnen und Umwelt
herbeiführen könnten und sich die Einflussfaktoren auf das System am Computer nachahmen lassen \cite{Britannica2023}. 
Eine Computersimulation kann auch dann genutzt werden, wenn das Erstellen eines realen Modells nicht möglich oder nicht rentabel ist. Ein weiterer Anwendungsfall einer Computersimulation tritt ein, wenn das reale Modell noch in der Entwicklungsphase ist. 
In diesen Fällen stellt die Simulation sicher, dass erste Versuche mit Algorithmen, die unabhängig vom Modell funktionieren, möglich sind. 
Dadurch kann damit begonnen werden an Technologien und Methodiken zu arbeiten, ohne auf eine reale Umsetzung warten zu müssen.
\subsection{Simulation im Kontext eines Ausweichalgorithmus}
Im Rahmen dieser Arbeit hat die Implementierung einer Simulation mehrere Vorteile. 
Es wird parallel an der Entwicklung des Autos und der zugehörigen Software gearbeitet.

Da die Implementierung der Software von Grund auf neu gestartet wird, 
wäre ein Warten auf die Fertigstellung des Autos aud zeitlichen Gründen nicht möglich.
Die Situation ist also eine der Situationen die in \ref{Simulation} beschrieben sind, 
in denen der Einsatz einer Simulation sinnvoll ist.

Ein weiterer Grund für den Einsatz einer Simulation ist ebenfalls in \ref{Simulation} beschrieben ist, 
ist ein möglicher Schadensfall. Das Auto ist von den Dimensionen nicht ausreichend groß, 
um einem Menschen zu verletzten, daher wären bei der Nutzung des realen Fahrzeuges kein Personenschaden zu befürchten.
Die Problematik in diesem Kontext ist die Anfälligkeit des Autos. 
Das Auto ist so konstruieert, das es mit wenig Leistung auskommt und nur die Elemente verbaut sind, 
dass es fahren kann. Aus diesem Grund wurden auf schützende Anbauteile wie Stoßdämpfer oder ähnliches verzichtet.
Aus diesem Grund könnte eine Kollision des Fahrzeuges mit einem Hindernis problematisch. 
Eine Kollision könnte Schäden am Fahrzeug veruursachen, die nur aufwendig, oder eventuell gar nicht repariert werden können.
Da die Algorithmik in frühen Entwicklungsstufen kritsiche, noch unerkannte Fehler beinahlten kann,
wäre es ein unnötiges Risiko die Software direkt auf dem Fahrzeuge auszuprobieren.

Ein weiterer Grund der für den Einsatz einer Simulation spricht, ist die Abhängigkeit vom Entwicklungsfortschritt 
des Autos. Sollte es dazu kommen, dass das Auto nicht rechtzeitig zur Verfügung steht, 
kann die Algorithmik zumindest im Rahmen der Simulation ausprobiert werden.

\subsection{Aufbau der Simulation}
Um die Simulation für eine erste Validierung der Algorithmik nutzen zu können, ist die Voraussetzung dass die simulierte Funktionalität auch der Funktionalität entspricht, die später im realen Einsatz genutzt wird.
Gibt es Abweichungen zwischen den Funktionalitäten, so kann nur eine Schätzung vorgenommen werden, wie sich der Algorithmus im praktischen Einsatz verhält. \\
Wie zu Beginn der Arbeit formuliert, soll zunächst ein Algorithmus entwickelt werden, der das Ausweichen auf gerader Linie ermöglicht. 
Da es aktuell keine Möglichkeit gibt, ein Ziel einzugeben, wird die aktuelle Ausrichtung des Fahrzeugs als Fahrtrichtung verwendet. Das ermöglicht eine Funktionalität unabhängig von der Position des Fahrzeugs. 
Dann fährt das Auto in diese Richtung. Wird auf der Strecke ein Hinderniss erkannt, versucht der Algorithmus dem Hinderniss auszueichen. Ist dies möglich wird das Auto auf die ursprüngliche Gerade zurückgeführt. \\
Um eine Simulation zu implenentieren, ist zu klären, ob diese Funktionalität in einer Simulation realitätsnah möglich ist, oder ob eine vereinfachte Version simuliert werden muss.\\

\subsubsection{Kritische Funktionalitäten}
Zunächst müssend die Funktionalitäten identifiziert werden, die in der Simulation Probleme veruursachen könnten.
\begin{enumerate}
    \item Sensordaten
    \item Umgebung 
    \item Ausweichen
    \item Lokalisierung
    \item Fahrzeug
\end{enumerate}

\paragraph{Sensordaten}
Die Sensordaten bilden die Grundlage für die gesamte Simulation. Auf den Sensordaten basiert die Lokalisierung im Raum und das Erkennen und Ausweichen eines Hindernisses. 
Die Simulation dieser Daten stellt damit die größte Herausforderung in der Simulation dar, da die Daten in Frequenz und Aufbau den realen Daten möglichst genau entsprechen sollten.
Vor allem der Aufbau der Daten sollte den realen Daten so nahe wie möglich kommen, da zusätzliche oder fehlende Daten in der Qualität der Auswertung deutlich zu erkennen sein könnten.
Gibt es Unterschiede in der Frequenz sind die Auswirkungen weniger problematisch. Eine geringere ode höhere Frequenz kann durch die mögliche Geschwindigkeiten des Modell-Autos oder an anderen Stellen ausgeglichen werden.

\paragraph{Umgebung}
Die Umgebung ist ebenfalls ein essentieller Bestandteil. Denn die Umgebung muss so simuliert werden, dass diese von den Sensoren erkannt werden kann. Ist das nicht der Fall, ist jede Simulation der Sensorik unbrauchbar.
Die genaue Implementierung ist unabhängig von der realen Welt. Dennoch ist darauf zu achten, dass die für die Sensorik notwendigen vorhanden sind.

\paragraph{Ausweichen}
Die Logik des Ausweichens ist die eigentliche Funktionalität die implenentiert werden soll. Diese ist der zentrale Baustein im System. 
Ist diese nicht vorhanden, ist die Simulation nicht brauchbar, um die Funktionalität zu validieren.

\paragraph{Lokalisierung}
Die Lokalisierung des simulierten Fahrzeugs ist ebenfalls ein wichitger Bestandteil der Simulation, denn die Lokalisierung ist dafür verantwortlich zu entscheiden ob das Ausweichmanöver abggeschlossen ist.

\paragraph{Fahrzeug}
Das Fahrzeug dient in der Simulation nur der Visualisierung. Physische Eigenschaften wie zum Beispiel der Kurvenradius und Reibung der Reifen sind nicht von Bedeutung. Außerdem kann das Auto visuell stark vereinfacht dargestellt werden. 


\subsubsection{Prüfung der kritischen Funktionalitäten}
\label{pr}

In diesem Abschnitt werden die einzelnen Funktionalitäten auf Umsetzbarkeit geprüft. Ist eine Umsetzung möglich, kann diese Funktionalität so implenentiert werden, andernfalls muss eine Alternative erarbeitet werden.

\paragraph{Sensordaten}
Folgende Sensordaten müssen simuliert werden: \\
- LIDAR: Der Lidar Sensor erzeugt durch Rotation und das Aussenden von Laserstrahlen eine Punktwolke die die Umgebung abbildet.
 Diese Daten sind gut zu simulieren, da hierfür auseghend von der aktuellen Position in jede Richtung Strahlen simuliert werden können, die die Entfernung von Objekten bestimmen. \\
- Ultraschall: Bei der Implementierung der Simulation des Ultraschall-Sensors ist mehr Aufwand nötig, da der Sensor nicht in jede Richtung Signale sendet, sondern konstant in eine Richtung. +
Die Komplexität kommt durch das Verhalten der Schallwellen, die sich im Raum ausbreiten. Trotz der zusätzlichen Komplexität sollte eine ausreichend gute Simulation möglich sein.\\
- Geschwindigkeit: Die Komplexität der Ermittlung der Geschwindigkeit ist gering, da diese Information als Variable direkt im simulierten Auto hinterlegt werden kann.\\
- Lenkwinkel: Kann ähnlich wie die Geschwindigkeit als Variable gespeichert werden. 

\paragraph{Umgebung}
Die Komplexität einer vereinfachten Umgebung ist als gering einzuschätzen, da für die Erkennung die Positionsdaten, sowie geometrische Form und Längen- und Breitenmaße ausreichend sind, um eine Überschneidung mit Strahlen oder Wellen mathematisch zu berechnen.
Die vereinfachte Umgebung enthält nur Gegenstände, deren Form eine einfache Berechnung zulässt. Dazu gehören zum Beispiel Quadrate und Rechtecke. \\
Soll die Komplexität der Umgebung erhöht werden, steigt der Aufwand der mathematischen Berechnungen. Die Komplexität der Simulation sollte aber auf einem ähnlichen Niveau bleiebn.

\paragraph{Ausweichen}
Das Ausweichen weist eine geringe Komplexität auf, da es für diese Aufgabe bereits viele verschiedene Lösungsstrategien gibt. 

\paragraph{Lokalisierung}
Die Lokalisierung erfolgt auf Basis der empfangenen Sensordaten. Die Korrektheit ist abhängig von der Qualität der enpfangenen Daten.
Aus diesem Grund ist an dieser Stelle mit der größten Abweichung von Simulation und Realität zu rechnen. Daraus resultiert die HErausforderung die Position an so vielen Parametern wie möglich zu bestimmen.
Denn je mehr Parameter genutzt werden, desto geringer der Einfluss von verfälschnten Daten. 

\paragraph{Fahrzeug}
Das Fahrzeug ist wenig komplex in der Implementierung, da eine vereinfachte Visualisierung, zum Beispiel mit einem Punkt, der sich bewegen kann, ausreichend ist.

\subsubsection{Auswertung der Prüfung}
Auf Basis der vermuteten Umsetzbarkeit der kritischen Funktionalitäten in Kapitel Prüfung \ref{pr} stellt sich die Simulation des Ausweichalgorithmus und der dazugehörigen Komponenten als machbar heraus. \\
Es gibt Teilaufgaben in der Simulation deren Komplexität höher ist, als die anderer Bestandteile, aber auch diese befinden sich in einem Umfang der umsetzbar ist.\\


\newpage
