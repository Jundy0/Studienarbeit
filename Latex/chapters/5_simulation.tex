\section{Simulation des Ausweichalgorithmus}

In diesem Kapitel wird beschrieben, wwarum eine Simulation hilfreich für die Entwicklung eines 
sicherheitskritischen Algorithmus, wie zum Beispiel ein Ausweichalgorithmus, sein kann.
Ausßerdem wird beschrieben, welche Aspekte der Simulation relevant für den Übertrag der Ergebnisse auf die Realität sind.

 \subsection{Was ist eine Simulation?}
 \label{Simulation}
 Bevor damit begonnen werden kann die verschiedenen Aspekte einer Simulation zu beleuchten, ist zu klären, was eine Simulation ist.
 Nach der Aussage von A. Maria ist eine Simulation eine Ausführung eines Modells eines Systems \cite{maria1997introduction}[p. 1, ch. 2].
Der Begriff des Modells wird ebenfalls in der Arbeit beschrieben. Ein Modell ist eine vereinfachte, funktionierende Repräsentation des Systems, das betrachtet werden soll \cite{maria1997introduction}[p. 1, ch. 1].\\
In der Simulationstechnik gibt es unterschiedliche Arten von Simulation. In diesem Kontext von Bedeutung ist die Unterscheidung zwischen realer Simulation und Computersimulation. Reale Simulationen kommen zum Einsatz, wenn durch einen Fehler keine Gefahr für Personen und Umwelt besteht.
Außerdem kann es sein, dass ein Nachstellen der Umweltbedingungen so komplex ist, dass eine Nachbildung am Computer nicht ausreichend möglich oder zeitlich zu aufwendig ist. Computersimulationen kommen dann zum Einsatz, wenn ein Fehler schädliche Folgen für Perosnen und Umwelt
herbeiführen könnten und sich die Einflussfaktoren auf das System am Computer nachahmen lassen \cite{Britannica2023}. 
Eine Computersimulation kann auch dann genutzt werden, wenn das Erstellen eines realen Modells nicht möglich oder nicht rentabel ist. Ein weiterer Anwendungsfall einer Computersimulation tritt ein, wenn das reale Modell noch in der Entwicklungsphase ist. 
In diesen Fällen stellt die Simulation sicher, dass erste Versuche mit Algorithmen, die unabhängig vom Modell funktionieren, möglich sind. 
Dadurch kann damit begonnen werden an Technologien und Methodiken zu arbeiten, ohne auf eine reale Umsetzung warten zu müssen.
\subsection{Simulation im Kontext eines Ausweichalgorithmus}
Im Rahmen dieser Arbeit hat die Implementierung einer Simulation mehrere Vorteile. 
Es wird parallel an der Entwicklung des Autos und der zugehörigen Software gearbeitet.

Da die Implementierung der Software von Grund auf neu gestartet wird, 
wäre ein Warten auf die Fertigstellung des Autos aud zeitlichen Gründen nicht möglich.
Die Situation ist also eine der Situationen die in \ref{Simulation} beschrieben sind, 
in denen der Einsatz einer Simulation sinnvoll ist.

Ein weiterer Grund für den Einsatz einer Simulation ist ebenfalls in \ref{Simulation} beschrieben ist, 
ist ein möglicher Schadensfall. Das Auto ist von den Dimensionen nicht ausreichend groß, 
um einem Menschen zu verletzten, daher wären bei der Nutzung des realen Fahrzeuges kein Personenschaden zu befürchten.
Die Problematik in diesem Kontext ist die Anfälligkeit des Autos. 
Das Auto ist so konstruieert, das es mit wenig Leistung auskommt und nur die Elemente verbaut sind, 
dass es fahren kann. Aus diesem Grund wurden auf schützende Anbauteile wie Stoßdämpfer oder ähnliches verzichtet.
Aus diesem Grund könnte eine Kollision des Fahrzeuges mit einem Hindernis problematisch. 
Eine Kollision könnte Schäden am Fahrzeug veruursachen, die nur aufwendig, oder eventuell gar nicht repariert werden können.
Da die Algorithmik in frühen Entwicklungsstufen kritsiche, noch unerkannte Fehler beinahlten kann,
wäre es ein unnötiges Risiko die Software direkt auf dem Fahrzeuge auszuprobieren.

Ein weiterer Grund der für den Einsatz einer Simulation spricht, ist die Abhängigkeit vom Entwicklungsfortschritt 
des Autos. Sollte es dazu kommen, dass das Auto nicht rechtzeitig zur Verfügung steht, 
kann die Algorithmik zumindest mit Hilfe der Simulation ausprobiert werden.

\subsection{Aufbau der Simulation}
Um die Ergebnisse und Erfahrungen der Simulation nutzen zu können, ist es wichtig, dass die Inhalte der Simulation
möglichst nah an die Realität herankommen.
Jede vorhandene Abweichung resultiert in einem weiteren Unsicherheitsfaktor. 
Ziel ist es für den Anfang einen Algorithmus zu simulieren, der basierend auf der aktuellen Position des Fahrzeuges eine Route zu einem Zielpunkt zu berechnet.
Die Route soll um die erkannten Hindernissen herumführen.
Zusätzlich sollen Eigenschaften des Fahrzeuges, wie der maximale Lenkwinkel, berücksichtigt werden, um eine möglichst realitätsnahe Route zu erhalten.   

Um eine Simulation zu implenentieren, ist zu klären, ob diese Funktionalität in einer Simulation realitätsnah möglich ist,
 oder ob eine vereinfachte Version simuliert werden muss.

\subsubsection{Kritische Funktionalitäten}
Zunächst müssend die Funktionalitäten identifiziert werden, die in der Simulation Probleme veruursachen könnten.
\begin{enumerate}
    \item Sensordaten
    \item Umgebung 
    \item Ausweichen
    \item Lokalisierung
    \item Fahrzeug
\end{enumerate}

\paragraph{Sensordaten}
Die Sensordaten bilden die Grundlage für die gesamte Simulation. 
Auf den Sensordaten basiert die Lokalisierung im Raum und das Erkennen und Ausweichen eines Hindernisses. 
Die Simulation dieser Daten stellt damit die größte Herausforderung in der Simulation dar, 
da die Daten in Frequenz und Aufbau den realen Daten möglichst genau entsprechen sollten.
Vor allem der Aufbau der Daten sollte den realen Daten so nahe wie möglich kommen, 
da zusätzliche oder fehlende Daten in der Qualität der Auswertung deutlich zu erkennen sein könnten. 
Außerdem bedeutet eine Abweichung in der Datenstruktur eine notwendige Anpassung der Implementierung, 
die nicht notwendig wäre, wenn die Datenstruktur übereinstimmen würde.
Gibt es Unterschiede in der Frequenz sind die Auswirkungen weniger problematisch. 
Ist die Frequenz langsamer als in der Simulation, kann dies durch eine langsamere Geschwindigkeit des Fahrzeuges kompensiert werden.
Eine FRequenz, die die Geschwindigkeit der Berechnungen überschreitet, kann durch das auslassen von einzelnen Scans kompensiert werden.
Die Bewegung des Fahrzeuges zwischen zwei Scans ist so gering, dass das Ignorieren von zum Beispiel jedem zweiten Scan,
kaum einen Einfluss auf das Ergebnis des Algorithmus haben sollte. 

Die genauen Auswirkungen von Abweichungen der Frequenz sind nicht bekannt, weswgen hier eine genaue Analyse notwendig wäre.

\paragraph{Umgebung}
Die Umgebung ist ebenfalls ein essentieller Bestandteil. 
Denn die Umgebung muss so simuliert werden, dass diese von den Sensoren erkannt werden kann. 
Ist das nicht der Fall, ist jede Simulation der Sensorik unbrauchbar, 
da dann keine Daten für den Ausweichalgorithmus zur Verfügung stehen und dann auch der simulierte LiDAR keine validen Daten liefert.
Eine 2D-Simulation der Umgebung ist ausreichend, da der LiDAR 2D-Daten liefert.
Unter der Voraussetzung der validen Datenerzeugung auf Basis der simulierten Umgebung, 
ist die genaue Implementierung der Umgebung nicht von Bedeutung. 

\paragraph{Ausweichen}
Das Ausweichen ist der zentrale Bestandteil der Software.
Die Implementierung in der Simulation soll auch in der Implementierung für die Steuerung des realen Autos zum Einsatz kommen.
Der Algorithmus selbst wird nicht simuliert, aber seine Daten kommen aus der Simulation. 
Außerdem wird der Output des Algorithmus in der Simulation Visualisiert. 
Daher ist es notwendig, den Output in einer Form zu generieren, dass er in der Simulation Visualisiert werdem kann.
Zum Output gehört die aktuelle Fahrzeugposition und der berechnete Weg zum Ziel.

Das Ziel der Visualisierung ist eine optische Validierung ob der berechnete Weg tatsächlich um die Hindernisse führt. 
Neben der Validierung des Weges kann auch die Berechnung der Fahrzeugposition in Ansätzen validiert werden, da erkennbar wird,
ob die neue Position ungefähr dem erwarteten Wert entspricht. Die genaue Position kann durch die Visualisierung alleine nicht validiert werdem.


\paragraph{Lokalisierung}
Die Simulation der Lokalisierung ist ebenfalls ein wichtiger Bestandteil der Simulation.
Vor allem, um einen verlässlichen Input für den Ausweichalgorithmus zu erhalten. 
Der Ausweichalgorithmus kann nur korrekt arbeiten, wenn die aktuelle Position des Fahrzeuges ausreichend genau bestimmt werdem kann.
Da die Lokalisierung in einer unbekannten Umgebung eine große Herausforderung ist, 
kann die Simulation genutzt werden, um zuverlässige Daten für den Algorithmus zu bekommen. 
Dies ist möglich, da die genaue Fahrzeugposition anhand der erfassten Steurerbefehle für das simulierte Fahrzeug bestimmt werden kann. 

\paragraph{Fahrzeug}
Das Fahrzeug dient in der Simulation nur der Visualisierung. 
Die Eigenschaften wie der maximale Lenkwinkel des Fahrzeuges werden für den Input des Ausweichalgorithmus benötigt, 
können aber direkt als Paramter des Algorithmus implenentiert werdem.
Die visuelle Darstellung des Fahrzeuges ist 
davon nicht betroffen und kann daher stark vereinfacht werden. 

\subsubsection{Prüfung der kritischen Funktionalitäten}
\label{pr}

In diesem Abschnitt werden die einzelnen Funktionalitäten auf Umsetzbarkeit geprüft. Ist eine Umsetzung möglich, kann diese Funktionalität so in der Simulation implenentiert werden, andernfalls muss eine Alternative erarbeitet werden.

\paragraph{Sensordaten}
Da aktuell nur die LiDAR-Daten genutzt werden, müssen auch nur die Daten dieses Sensors simuliert werden.
Der LiDAR rotiert um 360° und sendet in bestimmten Abständen Strahlen aus. 
Die Höhe der gesendeten Strahlen entspricht der Höhe des LiDAR-Sensors.
Wie in \ref{LiDAR-Daten} beschrieben, werden Hindernisse in einer Distanz zwischen 0.15 - 12 Metern identifiziert. 
Das Datenformat und die Frequenz der generierten Daten können dem Datenblatt \cite{Slamtec2020} entnommen werden.

Die Frequenz kann entsprechend simuliert werden und die Daten entsprechend dem Datenblatt genreriert werden. 
Um den Distanzbereich des realen Sensors zu simulieren ist eine entsprechende Skalierung der Simulation notwendig.
Die Distanzen vom Fahrzeug bis zu den Hindernissen können über etablierte Algotithmen, wie zum Beispiel Ray-Casting, 
oder eine angepasste Version dieser Algotithmen realisiert werden.

Die Sensordaten können also gut simuliert werdem, so dass keine Probleme entstehen sollten.

\paragraph{Umgebung}
Die Komplexität einer simulierten Umgebung ist als gering einzuschätzen.
Die Umgebung muss lediglich so implementiert werden, dass basiernd darauf korrekte Sensor-Daten genreriert werdem können.
Alle anderen Details der Implementierung für die Umgebung können stark vereinfacht werden, so dass eine Umsetzung problemlos möglich sein sollte.

\paragraph{Ausweichen}
Das Ausweichen um Hindernisse lässt sich in einer bekannten Umgebung mit gegebenem Ziel abstrahieren. 
Die Abstraktion an dieser Stelle ergibt einen Path-Finding Algorithmus. 
Für diese Art von Algorithmen gibt es bereits viele Lösungen die unterschiedliche Stärken haben.
Basierend auf den bereits existierenden Lösungen kann ein Ausweichalgorithmus mit den genannten Einschränkungen ohne Probleme implementiert werdem. 
 

\paragraph{Lokalisierung}
Die Lokalisierung stellt die größte Herausforderung dar.
Die Schwierigkeit der Lokalisierung liegt in der fehlenden Möglichkeit ein globales, absolutes Ortungssystem zu verwenden.
Dadurch erfolgt die Lokalisierung anhand der Sensordaten und der Anwendung eines Algorithmus der die relative Bewegung zwischen 
zwei Scans berechnet. 
Auf Grund von Performanceanfordeerungen ist es nicht möglich zu jeder Zeit eine perfekte Bewegung zu berechnen. 
Deshalb ist mit Abweichungen zu rechnen, die einen additiven Fehler in der berechneten Position veruursachen.
Die Umsetzung der Lokalisierung ist daher der kritische Punkt in der Entwicklung. 
Die Schwierigkeit in diesem Bereich beschränkt sich aber nicht auf die Simulation, sondern ist eine allgemeine Schwierigkeit in der Entwicklung einer solchen Algorithmik.


\paragraph{Fahrzeug}
Da die Simulation des Fahrzeuges unabhängig von den physikalischen Eigenschaften ist, ist dieser Teil der Simulation
wenig komplex und kann problemlos umgesetzt werdem. 

\subsubsection{Auswertung der Prüfung}
Basierend auf den einzelnen Teilbereichen der Simulation ergibt sich die Einschätzung, 
dass die Simulation ein sinnvolles und umsetzbares Mittel in der Entwicklung einer solchen Algorithmik ist.
Die größte Schwierigkeit resultiert aus der Problematik der relativen Berechnung der Lokalisierung. 
In den frühern Entwicklungsstufen lässt sich dieses Problem noch umegehen, in dem man auf die bekannte Position durch den 
Input der Steurungsbefehle zurückgreift. So bald aber die Grundlagen der Algorithmik für das Ausweichen um Hindernisse gelegt sind, 
muss eine Lösung gefunden werden, wie eine Lokalisierung basierend auf den realen, bzw. simulierten, Sensordaten umgesetzt werden kann.
  
\newpage
