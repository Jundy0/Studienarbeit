\section{Systemvoraussetzungen}
In diesem Kapitel wird definiert welche Voraussetzungen erfüllt sein müssen, um eine korrekte Funktion der Software sicherzustellen.\\
Die Algorithmik für die Ortung des Fahrzeuges befindet sich in einem frühen Entwicklungsstadium. Aus diesem Grund müssen einige Bedingugnen eingehalten werden. 
\paragraph{Umweltvoraussetzungen}
\begin{itemize}
    \item[1)] Das Fahrzeug darf nur in einer statischen Umgebung autonom gefahren werden. Die Objekte in der Umgebung dürfen während der Fahrt nicht bewegt werdem.Grund dafür ist, dass nicht bekannt ist, wie sich eine dynamsiche Umgebung auf die Präzision der Lokalisierungsalgorithmik auswirkt. 
    \item[2)] Laut Datenblatt \cite{Slamtec2020} liegen die Distanzen die der Sensor erfassen kann, zwischen 0.15 - 12 Metern. Um die eine Versorgung mit validen Daten sicherzustellen, muss die Umgebung so gebaut sein, dass zu jedem Moment von mindestens drei Seiten Objekte in einem Umkreis von maximal 10 Metern vorhanden sind, so dass der Lidar ausreichend viele, korrekt Werte liefert.  Hintergrund der Beschränkung sind ebenfalls die unbekannten Auswirkungen auf die Lokalisierungsalgorithmik.
    \item[3)] Stichpunkt drei
\end{itemize}

