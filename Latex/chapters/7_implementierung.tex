\section{Implementierung}

\subsection{Aufbau der Implementierung}

Der gesamte Aufbau der Implementierung ist in drei Projekte aufgeteilt: Core, Simulation und Lidar. 
Das Core-Projekt ist eine Library, welche keine ausführbare Datei und lediglich die Implementierungen der Algorithem bzw. die Logik für das Steuern und Ausweichen des Fahrzeug enthält. 
Das Simulation-Projekt dient für die Simulation des autonomen Fahrzeugs und die Implementierungen der Algorithem. 
Das Lidar-Projekt enthählt den Code, welcher auf das eingentliche Fahrzeug, bzw. den Raspberry Pi des Fahrzeugs, geladen und auf diesem ausgeführt wird. 

Nachfolgend werden die einzelnen Projekte näher beschrieben. 

\subsubsection{Core-Projekt}

Wie bereits beschreiben enthält das Core-Projekt die Implementierungen der verwendeten Algorithmen und die Logik zum Steuern des Fahrzeugs. 
Um auf Daten von Sensoren, sowie die Steuerung des Autos zuzugreifen, werden für diese Interfaces verwendet, über welche im Core-Projekt auf diese Zugegriffen werden kann. 
Dies hat außerdem noch den vorteil, dass die Implementierung der Interfaces ausgetauscht werden kann, 
damit nicht die eingentlichen Senosren ausgelesen und das Auto gesteuert wird, sondern nur mit einer Simulation interagiert wird. 
Damit bei der Verwendung des Core-Projekt in einem der anderen Projekt die richtige Implementierung für die jeweiligen Interfaces verwendet werden, 
sind diese zum einen nur in dem jeweiligen Projekt enthalten und werden zum anderen bei dem Start des Programms an das Core-Projekt übergeben. 
Letzteres geschieht nach dem Dependency Injection Prinzp (TODO: Quelle und beschreibung DI).

\subsubsection{Simulation-Projekt}

\subsubsection{Lidar-Projekt}



\LARGE\textbf{!!ToDo: Sehr viel fehlt noch. Kommt bald!!}\normalsize

\newpage
