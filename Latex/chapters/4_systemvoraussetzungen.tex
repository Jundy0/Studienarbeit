\section{Systemvoraussetzungen und Ein\-schränkungen}
In diesem Kapitel wird definiert welche Voraussetzungen erfüllt sein müssen, um eine korrekte Funktion der Software sicherzustellen.
Die Algorithmik für die Ortung des Fahrzeuges befindet sich in einem frühen Entwicklungsstadium. 
Aus diesem Grund müssen einige Bedingugnen eingehalten werden. 

\subsection{Umweltvoraussetzungen}
In diesem Abschnitt werden die Voraussetzungen an das Einsatzgebiet des Autos definiert.

\begin{enumerate}[leftmargin=*]
    \item \textbf{Statische Umgebung} \\
    Das Fahrzeug darf nur in einer statischen Umgebung autonom gefahren werden. 
    Die Objekte in der Umgebung dürfen während der Fahrt nicht bewegt werdem.
    Grund dafür ist, dass nicht bekannt ist, wie sich eine dynamsiche Umgebung auf die Präzision der Lokalisierungsalgorithmik auswirkt. 
    
    \item \textbf{Hindernisse} \\
    Laut Datenblatt \cite{Slamtec2020} liegen die Distanzen die der Sensor erfassen kann, zwischen 0.15 - 12 Metern.
    Um die Versorgung mit validen Daten sicherzustellen, muss die Umgebung so gebaut sein, 
    dass zu jedem Zeitpunkt, sowohl in x-Richtung als auch in y-Richtung, Objekte mit einem Maximalen Abstand von maximal 10 Meteren vorhanden sind.
    Zudem müssen die Hindernisse die Scan-Ebene des \ac{lidar}-Sensor schneiden, da sie ansonsten nicht erkannt werden.
    Auch sollte eine Mindestanzahl von 100 Punkten pro Scan geliefert werden.
    Bei Nichteinhaltung der genannten Bedingungen ist eine Lokalisierung mittels des implementierten Algorithmus ungenau.
    
    \item \textbf{Trockene Umgebung} \\
    Das Fahrzeug darf nur in einem vor Wasser geschützten Bereich verwendet werden.
    Hintergrund ist dass die Elektronik nicht vor eindringendem Wasser geschützt ist.
    Eindringendes Wasser könnte das Fahrzeug so beschädigen, dass es nicht mehr funktioniert.
   
    \item \textbf{Höhe der Hindernisse} \\
    Der verwendete LiDAR-Sensor liefert 2D-Daten in einem 360 Grad Winkel. 
    Das bedeutet, dass die Hindernisse auf der Höhe des LiDAR-Sensors sein müssen um erfasst zu werden.
    Sind die Hindernisse unterhalb des LiDAR, so können diese Hindernisse von diesem nicht erfasst werden.
    Dadurch können diese auch nicht um Ausweichalgorithmus berücksichtigt werden, was ein Ausweichen dieser Hindernisse unmöglich macht.
\end{enumerate}

\subsection{Einschränkungen}
In diesem Abschnitt werden die Voraussetzungen und Einschränkungen beschrieben, 
die berücksichtigt werden müssen, wenn die Algorithmik Weiterentwickelt wird.

\begin{enumerate}[leftmargin=*]
    \item \textbf{Steuerung} \\
    Der Algorithmus muss auf die Steuerungsmöglichkeiten des Autos angepasst sein.
    Das bedeutet, dass der Algorithmus vor allem den Lenkwinkel und die Breite des Autos berücksichtigen muss.
    Die Steuerung selbst soll über eine klar definierte Schnittstelle erfolgen.
    Eine Nutzung des Algorithmus für ein anderes Auto wird daher nur nach einer Anpassung der Algorithmik möglich sein.

    \item \textbf{Simulation} \\
    Die Simulation soll einen ersten Ansatz für das Testen und die Visualisierung bieten. 
    Deshalb ist die implementierung eine vereinfachte Darstellung eine Umgebung.
    Die Simulation soll eine einfache Top-Dowm Perspektive auf das Auto und die Umgebung bieten. 
    Das virteulle Auto soll manuell und mittels Algorithmus gesteuert werden können.
    Auch hier gilt zu beachten, dass die Steuerung des simulierten Autos möglichst identisch mit der des eigentlichen Autos ist.
    Das gilt auch für die Schnittstelle zur Steuerung, welche der Algorithmus nutzen wird.
    Auf Grund der Abhängigkeit des Algorithus von den konkreten Werten des Autos, 
    wird die Simulation bezogen auf die Ansteuerung des Fahrzeuges und der Visualisierung der Umgebung 
    keine exakte, detaillierte Abbildung der Realität bieten. Das tatsächliche Verhalten eines verwendeten Fahrzeuges 
    kann daher von der Simulation abweichen. 

    \item \textbf{Laufzeit} \\
    Der Laufzeit des Algorithmus soll ausreichend kurz sein.
    Das bedeutet, dass Hindernisse in unter einer halben Sekunde erkannt werden und entsprechend reagiert werden soll.
    Dieser Wert ist kein empirisch, oder anders wissenschaftlich validierter Wert. 
    Für den Einsatz in der Realität muss ein fundierter Wert ermittelt werden, 
    so dass die tatsächlich benötigte Geschwindigkeit berücksichtigt werdem kann.
\end{enumerate}

\newpage