\section{Überblick Hardware und Software}
In diesem Kapitel wird die Hardware und Software beschrieben, welche zur Bearbeitung der Studienarbeit zur Verfügung stehen.
Zusätzlich wird die Technologie-Entscheidung erläutert und begründet.

\subsection{Hardware}
\label{hardware_uebersicht}
Dieser Abschnitt beschreibt die Hardware, welche für die Entwicklung des Ausweichalgorithmus relevant ist. 

\begin{enumerate}[leftmargin=*]
    \item \textbf{Raspberry Pi 4 Model B} \\
    Der Raspberry Pi 4 ist ein Single-Board-Computer, welcher im Jahr 2019 auf dem Markt erschien. 
    Er besitzt eine ARM-basierte 64-Bit CPU, welche mit 1.5GHz getaktet ist. Das Modell, welches im Rahmen unserer Studienarbeit genutzt wird, besitzt 8 GB Arbeitsspeicher. 
    Außerdem verfügt der Raspberry Pi 4 über 40 \acf{gpio} Pins, welche zur Kommunikation mit den Sensoren und der Steuerungs-Schnittstelle genutzt werden können. \cite{RasPi2024}

    \item \textbf{Slamtec RPLiDAR A1M8-R6} \\
    Der RPLiDAR A1M8-R6 von Slamtec ist ein zweidimensionaler Laser-Scanner, welcher mittels \acf{lidar}, ein 360° Scan der Umgebung erstellen kann. \cite[p. 3]{Slamtec2020}
    Er hat eine effektive Reichweite von 0.15 bis 12 Meter und bei einer Scan-Rate von 5.5 Scans pro Sekunde, sowie eine Scan-Frequenz von 8000 Hz, eine Auflösung von weniger als einem Grad. \cite[p. 8]{Slamtec2020}
    
    \item \textbf{Weitere Sensoren} \\
    Da der \ac{lidar}-Sensor nur zweidimensionale Scans macht, können Hindernisse, welche kleiner als die Scan-Höhe des \ac{lidar} sind, von diesem nicht erfasst werden. 
    Daher sind weitere Sensoren, wie z.B. Ultraschall-Sensoren notwendig, um auch niedrige Hindernisse erkennen zu können. 
    Außerdem wäre ein Sensor zur Bestimmung des aktuellen Lenkwinkels und ein weiterer Sensor zum Bestimmen der aktuellen Geschwindigkeit sinnvoll. 
    Die Daten dieser Sensoren könnten bei der Ermittlung der Position im Raum von Nutzen sein. 
\end{enumerate}

\subsection{Software}
In diesem Abschnitt wird auf die Software eingegangen, welche zur Entwicklung des Algorithmus zur Verfügung steht.

\begin{enumerate}[leftmargin=*]
    \item \textbf{\acf{ros}} \\
    \ac{ros} ist eine Ansammlung von Werkzeugen und Bibliotheken, wie Treiber und Algorithmen, welche bei der Entwicklung von Roboter-Anwendungen helfen sollen. 
    Hierbei ist ROS vollständig Open-Source und bietet zudem eine ausführliche Dokumentation, Foren und eine große Community. \cite{Ros2024}
    Des Weiteren bietet Slamtec, der Hersteller des zur Verfügung stehenden \ac{lidar}-Sensors, eine Bibliothek, zur Nutzung des \ac{lidar}-Sensors, in Kombination mit verschiedenen Versionen des \ac{ros} an. \cite{RplidarRos2023}

    \item \textbf{Slamtec RPLiDAR Public SDK} \\
    Slamtec bietet, neben der \ac{ros}-Bibliothek, auch eine öffentlich zugängliches \ac{sdk} für sämtliche RPLiDAR-Produkte an. 
    Das \ac{sdk} ist in C++ geschrieben und unter der BSD 2-clause Lizenz lizenziert. \cite{RplidarSDK2023}

    \item \textbf{Slamtec RPLiDAR SDK Python-Ports} \\
    Die Slamtec RPLidar Sensoren sind, aufgrund des erschwinglichen Preises, vor allem bei Einsteigern sehr beliebt.
    Auch die Programmiersprache Python ist in den letzten Jahren immer beliebter geworden.
    Da Slamtec selbst kein Python-\ac{sdk} anbietet, entstanden über die Jahre diverse Ports des C++-\ac{sdk}.
\end{enumerate}

\subsection{Technologie-Entscheidung}
Dieser Abschnitt erläutert die Entscheidungen für die diversen Technologien, welche zur Bearbeitung des praktischen Teils der Arbeit gewählt wurden.

\paragraph{Hardware} \mbox{}\\
Die Auswahlmöglichkeiten der Hardware sind sehr beschränkt.
Da die notwendigen Berechnungen einiges an Leistung benötigen, wird sich für das leistungsstärkste, verfügbare Modell des Raspberry Pi entschieden.
Der Raspberry Pi 4 Model B bietet neben einer 64-Bit CPU mit ausreichender Leistung, auch 8 GB Arbeitsspeicher und einen Formfaktor der klein genug ist, um eine Integration in das Fahrzeug zu ermöglichen.
Zusätzlich bietet er ausreichend Schnittstellen um mit den diversen Sensoren und Motoren kommunizieren zu können.

Als Hauptsensor für das Messen der Umgebung steht nur der RPLiDAR A1M8-R6 von Slamtec zur Verfügung.
Weitere Sensorik soll zwar auf dem Auto verbaut werden, jedoch erstmal nicht von der Software berücksichtigt werden.
Der Grund hierfür ist der Zeitpunkt, zu dem die Sensorik verbaut werden kann.
Die Verarbeitung der Daten des \ac{lidar} kann ohne Auto oder per Simulation getestet werden, wohingegen es bei der anderen Sensorik sehr stark auf die Integration in dem Fahrzeug ankommt.
Da diese jedoch erst zum Schluss in dem Fahrzeug verbaut werden, wäre eine Software-Integration dieser Sensorik nur schwer rechtzeitig umzusetzen.

\paragraph{Software} \mbox{}\\
Maßgeblich verantwortlich für die Auswahl der Programmiersprache, ist die Auswahl der Software, welche verwendet wird um den \ac{lidar} anzusteuern.

Die eine Möglichkeit wäre das Nutzen des offiziellen Slamtec RPLLiDAR \ac{ros}-Paket.
Bei ROS handelt es sich jedoch um eine sehr umfangreiche Software. 
Diese kommt mit vielen, für die Studienarbeit nicht relevanten, Komponenten daher.
Des Weiteren benötigt \ac{ros} ein anderes, nicht für den Raspberry Pi optimiertes, Betriebssystem wie Ubuntu.
Das hat zu Folge, das weitere Ressourcen für das Betriebssystem benötigt werden und nicht für die notwendigen Berechnungen zur Verfügung stehen.

Die Alternative zur Verwendung von \ac{ros}, ist das Nutzen eines \ac{sdk}.
Die Python-Ports des SDK sind alle schon einige Jahre alt und haben teilweise keine wirklich übersichtliche Struktur.
Das offizielle C++-\ac{sdk} hingegen wird immer noch regelmäßige upgedatet und bietet eine umfangreiche Dokumentation sowie einige Beispielprogramme.

Um die begrenzt vorhandenen Ressourcen des Raspberry Pi optimal nutzen zu können, ist daher die Nutzung des C++-\ac{sdk} und einem entsprechenden Interface die beste Lösung.
Da das \ac{sdk} in C++ geschrieben ist, ergibt es Sinn, die restliche Software ebenfalls in C++ zu implementieren.
Zusätzlich verbessert die Nutzung einer kompilierten Programmiersprache wie C++ die Laufzeit der Software was die Reaktionszeit des Autos verbessert.

\newpage
