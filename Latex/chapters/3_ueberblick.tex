\section{Überblick Hardware und Software}
In diesem Kapitel wird auf die

\subsection{Hardware}
Dieser Abschnitt beschreibt die Hardware, welche für die Entwicklung des Ausweichalgorithmus relevant ist. 

\begin{enumerate}
    \item \textbf{Raspberry Pi 4 Model B} \\
    Der Raspberry Pi 4 ist ein single-board-computer, welcher im Jahr 2019 auf dem Markt erschien. 
    Er besitzt eine ARM-basierte 64-bit CPU, welche mit 1.5GHz getaktet ist. Das Modell, welches im Rahmen unserer Studienarbeit genutzt wird, besitzt 4GB Arbeitsspeicher. 
    Außerdem verfügt der Raspberry Pi 4 über 40 \acf{gpio} Pins, welche zur Kommunikation mit den Sensoren und der Steuerungs-Schnittstelle genutzt werden können. \cite{RasPi2024}

    \item \textbf{Slamtec RPLiDAR A1M8-R6} \\
    Der RPliDAR A1M8-R6 von Slamtec ist ein zweidimensionaler Laser-Scanner, welcher mittels \acf{lidar}, ein 360° Scan der Umgebung erstellen kann. \cite[p. 3]{Slamtec2020}
    Er hat eine effektive Reichweite von 0.15 bis 12 Meter und bei einer Scan-Rate von 5.5 Scans pro Sekunde, sowie eine Scan-Frequenz von 8000 Hz, eine Auflösung von weniger als einem Grad. \cite[p. 8]{Slamtec2020}
    
    \item \textbf{Weitere Sensoren} \\
    Da der \ac{lidar}-Sensor nur zweidimensionale Scans macht, können Hindernisse, welche kleiner wie die Scan-Höhe des \ac{lidar} sind, von diesem nicht erfasst werden. 
    Daher sind weitere Sensoren, wie z.B. Ultraschall-Sensoren notwendig, um auch niedrige Hindernisse erkennen zu können. 
    Außerdem wäre eine Sensor zur Bestimmung des aktuellen Lenkwinkels und ein weiterer Sensor zum Bestimmen der aktuellen Geschwindigkeit sinnvoll. 
    Die Daten dieser Sensoren könnten bei der Ermittlung der Position im Raum von Nutzen sein. 
\end{enumerate}

\subsection{Software}
In diesem Abschnitt wird auf die Software eingegangen, welche zur Entwicklung des Algorithmus zur Verfügung steht.

\begin{enumerate}
    \item \textbf{\acf{ros}} \\
    \ac{ros} ist eine Ansammlung von Werkzeugen und Bibliotheken, wie Treiber und Algorithmen, welche bei der Entwicklung von Roboter-Anwendungen helfen sollen. Hierbei ist ROS vollständig Open-Source und bietet zudem eine ausführliche Dokumentation, Foren und eine große Community. \cite{Ros2024}
    Des Weiteren bietet Slamtec, der Hersteller des zur Verfügung stehenden \ac{lidar}-Sensors, eine Bibliothek, zur Nutzung des \ac{lidar}-Sensors, in Kombination mit verschiedenen Versionen des \ac{ros} an. \cite{RplidarRos2023}

    \item \textbf{Slamtec RPLIDAR Public SDK} \\
    Slamtec bietet, neben der \ac{ros}-Bibliothek, auch eine öffentlich zugängliche SDK für sämtliche RPLiDAR-Produkte an. Die SDK ist in C++ geschrieben und unter der BSD 2-clause Lizenz lizenziert. \cite{RplidarSDK2023}
\end{enumerate}

\subsection{Technologie-Entscheidung}

\newpage
