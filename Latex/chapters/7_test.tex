\section{Testen der Simulation und der realen implementierung}
In diesem Abschnitt geht es darum, wie Tests strukturiert und implementiert werden. Dabei geht es sowohl um Tests für die Simulation als auch um Tests für die reale Implementierung. \\
Bei einem selbstfahrenden Fahrzeug hanelt es sich um ein echtzeitkritischen System, bei dem Fehler zu schwerwiegenden Konsequenzen führen können. Daher ist es wichtig, dass die Software des Fahrzeugs ausgiebig getestet wird. \\
Funtioniert die Hindernisserkennung nicht innerhalb eines bestimmten Zeitraums, kann das Fahrzeug nicht rechtzeitig bremsen oder ausweichen und es kann zu einem Unfall komment, 
der Schäden am Fahrzeuge selbst, anderen Gegensänden oder auch an Personen verursachen kann. \\
Das Fahrzeug das im Rahmen dieser Arbeit entwickelt wird, ist zwar nicht für den Einsatz im realen Straßenverkehr vorgesehen und für diesen auch nicht zugelassen,
dennoch sollten die Systeme so sicher wie möglich entwickelt werden. \\
Um die abstrakte Formulierung 'so sicher wie möglich' zu konkretisieren ist es unumgänglich Metriken für die Performance der Software zu definieren.
\subsection{Grudlagen für das Testen von Software}
Das Testen von Software hat mehrere Ziele. Zum einen soll das Testen sicherstellen, dass die Software die Anforderungen erfüllt. 
Zum anderen soll das Testen dabei helfen Fehlerfälle, insbesondere Randfälle, identifizieren und beheben zu können. \\
Anders als erwartet werden könnte, dient das Testen von Software nicht dazu, die Fehlerfreiheit der Software zu beweisen, denn das ist nicht möglich. Bewiesen wurde das durch A. M. Turing. in seiner Publikation konnte er beweisen, dass es keine Maschine gibt, die die Fehlerfreiheit einer anderen Maschine garantiert \cite[S. 259ff]{maria1997introduction} \\
Übertragen auf Software bedeutet das, dass es keine Software gibt, die die Fehlerfreiheit einer anderen Software garantieren kann. \\
Um die Testfälle entwickeln zu können, ist es notwendig, die oben beschriebenen Anforderungen und Einschränkungen in Metriken zu übersetzen. Für diese Metriken können dann Testfälle entwickelt werden. \\
Auf Grund der vielen notwendigen Testfälle ist es hilfreich, die Testfälle für Simulation und reale Implementierung so ähnlich wie möglich zu halten. 
Wenn sich diese Testfälle ähneln, ist es einfacher, die Testfälle zu entwickeln und zu warten. Außerdem ist es einfacher, die Ergebnisse der Tests von Simulation und Realität zu vergleichen.\\
Dadurch lassen sich Rückschlüsse auf die Qualität der Simulation ziehen. Je ähnlicher die Tests der Simulation und der Realität sind, desto ähnlicher ist die Implementierung der Simulation und der Realität. 
Nur durch diese Ähnlichkeit ist es möglich, die Erkenntnisse der Simulation auf die Realität zu übertragen. Ansosnten wäre die simulation eine Art Spielplatz für das Entwickeln von Algorithmen. Auf diesem Spielplatz können erste Erfahrungen gesammelt werden, während für die Realität eine weitere Einarbeitung notwendig
wäre. \\
Im Optimalfall sind die Testfälle identisch. Das wird aber in der Praxis nicht möglich sein, da die Simulation immer eine gewisse Abstraktion der Realität darstellt. Durch die Abstraktion können Details verloren gehen, oder so abgewandelt werden, dass auch die Testfälle für diese Situationen angepasst werden müssen.



\subsection{Metriken und Anforderungen für die Performance der Software}
Metriken dienen dem Zweck die Ergebnisse eines Prozesses zu quantifizieren \cite[S. 204]{nirpal2011brief}. Durch die Quantifizierung der Ergebnisse können diese miteinander verglichen und ausgewertet werden. \\
Durch Metriken kann die Funktionalität der Software objektiv gemessen werden. \\
Um Metriken definieren zu können sind zunächst die Anforderungen bezügich der Performance zu definieren. \\
\subsubsection{Anforderungen an die Software}

Folgende Anforderungen sind allgemeingültig:\\
\begin{itemize}
    \item Die Software muss in der Lage sein, Hindernisse zu erkennen und darauf zu reagieren.
    \item Die Software muss in der Lage sein, die Position des Fahrzeugs zu bestimmen.
    \item Die Software muss in der Lage sein, die Fahrtrichtung des Fahrzeugs zu erkennen.
    \item Die Software muss in der Lage sein, die Geschwindigkeit des Fahrzeugs zu bestimmen.
\end{itemize}

Folgende Anforderungen sind speziell für die Simulation:\\
\begin{itemize}
    \item Simulations-Anforderungen
\end{itemize}
Folgende Anforderungen sind speziell für die reale Anwendung:\\
\begin{itemize}
    \item Realitätsspezifische-Anforderungen
\end{itemize}

\subsubsection{Metriken für die Anforderungen}
Hier werden die Metriken definiert

\subsection{Testkonzept}
Basierend auf den Metriken und Anforderungen kann ein Testkonzept erstllt werden, das befolgt wird um die Ergebnisse zu validieren.

