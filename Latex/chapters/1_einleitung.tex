\section{Einleitung}
%Die Automatisierung im Straßenverkehr befindet sich in stetigem Wachstum, wobei selbstfahrende Fahrzeuge zunehmend an Bedeutung gewinnen. \\
%Auch im Bereich der Modell-Autos und Roboter ist Automatisierung ein präsentes Thema. \\
%Zwar ist die Umsetzung eines selbstfahrenden Modell-Autos oder Roboters, aufgrund der vorhersehbaren und weniger komplexen Umgebung, 
%einfacher als die Umsetzung eines selbstfahrenden PKW, jedoch stellt sie trotzdem eine Herausforderung dar.

\textbf{Meine Idee}\\

Egal ob im Straßenverkehr, in der Robotik, oder in Modellversuchen, in vielen Bereichen gewinnt die Möglichkeit zur autonomen Fortbewegung an Bedeutung.
Autonome Fortbewegung verspricht viele Vorteile, die die Sicherheit und Effizienz erhöhen sollen.
Durch die Automatisierung soll das Risiko von menschliche Fehlern reduzieren und auch die Anwesenheit von Menschen an manchen Stellen überflüssig machen.
Die möglichen Anwendungsbereiche reichen von fahrerlosen Fahrzeugen bis hin zur autonomen Erkundung von Risikogebieten durch speziell ausgerüstete Fahrzeuge.

Die erhofften Vorteile solcher Technologien sorgen dafür, dass intensive Entwicklungsarbeit geleistet wird, um den Einsatz in der Praxis zu ermöglichen.
In der Entwicklung dieser Technologien gibt es einige Herausforderungen, die genau analysiert werden müssen, um eine passende Lösung zu finden.

In dieser Arbeit geht es darum, ein selbstfahrendes Fahrzeug zu entwickeln, dass sich autonom zu einem Zielort bewegen kann, auch wenn die Umgebung unbekannt ist.\\
\textbf{Bis hier bin ich zufrieden, aber die Überleitung zum Inhalt, passt noch nicht ganz.}

Um die Komplexität der Entwicklung zu verringern, werden Einschränkungen für die Umgebung definiert. 
Doch trotz dieser Einschränkungen stellt die Entwicklung einer solchen Technologie eine Herausforderung dar.


\textbf{Meine Idee + CHatGPT mit Anpassungen von mir.}\\
Egal ob im Straßenverkehr, in der Robotik oder in Modellversuchen, in vielen Bereichen gewinnt die Möglichkeit zur autonomen Fortbewegung an Bedeutung. 
Autonome Fortbewegung verspricht viele Vorteile, die die Sicherheit und Effizienz erhöhen sollen. 
Durch die Automatisierung soll das Risiko von menschlichen Fehlern reduziert und auch die Anwesenheit von Menschen an manchen Stellen überflüssig gemacht werden. 
Die möglichen Anwendungsbereiche reichen von fahrerlosen Fahrzeugen bis hin zur autonomen Erkundung von Risikogebieten durch speziell ausgerüstete Fahrzeuge.

Die erhofften Vorteile solcher Technologien sorgen dafür, dass intensive Entwicklungsarbeit geleistet wird, um den Einsatz in der Praxis zu ermöglichen. 
In der Entwicklung dieser Technologien gibt es einige Herausforderungen, die genau analysiert werden müssen, um eine passende Lösung zu finden. 
In dieser Arbeit geht es darum, ein selbstfahrendes Fahrzeug zu entwickeln, das sich autonom zu einem Zielort bewegen kann, auch wenn die Umgebung unbekannt ist.

Um die Komplexität der Entwicklung zu verringern, werden Einschränkungen für die Umgebung definiert. 
Trotz dieser Einschränkungen bleibt die Entwicklung solcher Technologien herausfordernd. 
Diese Herausforderungen umfassen die präzise Lokalisierung des Fahrzeugs, 
die zuverlässige Erkennung und Umgehung von Hindernissen sowie die Integration und Abstimmung der benötigten Hardware und Software.

In den folgenden Kapiteln wird detailliert auf die Problemstellung, die spezifischen Ziele dieser Arbeit und die geplanten Umsetzungsschritte eingegangen. 
Hierbei wird erläutert, welche technischen und methodischen Ansätze verfolgt werden, um die genannten Herausforderungen zu bewältigen und ein autonomes Fahrzeug erfolgreich zu entwickeln.