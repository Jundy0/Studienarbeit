\section{Einleitung}

Egal ob im Straßenverkehr, in der Robotik oder in Modellversuchen, in vielen Bereichen gewinnt die Möglichkeit zur autonomen Fortbewegung an Bedeutung. 
Autonome Fortbewegung soll die Sicherheit und Effizienz erhöhen. 
Durch die Automatisierung wird die Notwendigkeit menschlicher Anwesenheit überflüssig. 
Die möglichen Anwendungsbereiche reichen von fahrerlosen Fahrzeugen bis hin zur autonomen Erkundung von Risikogebieten durch speziell ausgerüstete Fahrzeuge.

Die erhofften Vorteile solcher Technologien sorgen dafür, dass intensiv in diesem Bereich entwickelt wird. 
In dieser Arbeit geht es darum, ein selbstfahrendes Modell-Fahrzeug zu entwickeln, das sich autonom zu einem Zielort bewegen kann, auch wenn die Umgebung unbekannt ist.

Herausforderungen hierbei umfassen die präzise Lokalisierung des Fahrzeugs, 
die zuverlässige Erkennung und Umgehung von Hindernissen sowie die Integration und Abstimmung der benötigten Hardware und Software.

In den folgenden Kapiteln wird detailliert auf die Problemstellung, die spezifischen Ziele dieser Arbeit und die Umsetzungsschritte eingegangen. 
Hierbei wird erläutert, welche technischen und methodischen Ansätze verfolgt werden, um die genannten Herausforderungen zu bewältigen und ein autonomes Fahrzeug erfolgreich zu entwickeln.