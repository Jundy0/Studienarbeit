\documentclass[12pt]{article}

\usepackage{amsmath}

\usepackage{graphicx}

\usepackage{hyperref}

\usepackage[utf8]{inputenc}
\usepackage[ngerman]{babel}

\title{My LaTeX Title}

\author{Guillaume Blanchet}

\date{2020–01–13}

\begin{document}

\pagenumbering{roman}

\begin{titlepage}
	\begin{minipage}{6in}
	\vspace*{-2cm}
    \centering
    \raisebox{-0.5\height}{}
    \hfill
    \raisebox{-0.5\height}{\includegraphics[height=2cm]{graphics/DHBW_logo.jpg}}
    \end{minipage}
	\begin{center}
		\vspace*{0.5cm}
		\LARGE\textbf{Weiterentwicklung eines selbstfahrenden Fahrzeuges mit Lidar
        und anderen Sensoren}\\
		% \Large\rm\mySubTopic\\
		\vspace*{2cm}
		\textbf{Studienarbeit}\\
		\normalsize
		{\"u}ber die ersten drei Quartale des 3. Studienjahres\\
		\vspace*{1.3cm}
		an der Fakult{\"a}t f{\"u}r Technik\\
		im Studiengang Informationstechnik\\
		\vspace*{1cm}
		an der DHBW Ravensburg\\
		Campus Friedrichshafen\\
		\vspace*{1cm}
		von\\
		Justin Serrer - 5577068 - TIT21 \\ 
		Timo Waibel - 8161449 - TIT21 \\
		Janik Frick - 4268671 - TIT21 \\
		\vspace*{2cm}
		\vfill
	\end{center}
	\begin{tabular}{ll}
	\end{tabular}
\end{titlepage}
\newpage
\tableofcontents
\newpage
\listoffigures
\newpage

\section{Einleitung}
In diesem Abschnitt wird auf das generelle Ziel der Arbeit eingegangen.
Des weiteren wird Erläutert, weshalb das Ziel der Arbeit wichtig ist, wie die Arbeit aufgebaut ist und welche Probleme und Schwierigkeiten durch bereits getätigte Versuche einer Umsetzung des Arbeits-Ziels bereits bekannt sind.

\begin{enumerate}
    \item Was ist das Ziel der Studienarbeit?\\
    Das Hauptziel der Arbeit ist die Entwicklung und Implementierung eines Ausweich-Algorithmus für ein selbstfahrendes Modell-Auto.
    Zur Umsetzung dieses Ziels, stehen zusätzlich zu einem RPLiDAR A1M8-R6 der Firma Slamtec, auch weitere Sensoren zur Verf{\"u}gung.
    Der Hardware-Teil der Arbeit wird von einer Gruppe an E-Technik-Studenten {\"u}bernommen, weshalb eine weitere Aufgabe die Koordination mit der anderen Gruppe ist, so dass eine Zusammenfh{\"u}rung von Hard- und Software ohne großen Aufwand m{\"o}glich ist.
    Da die Hardware nicht rund um die Uhr zur Verf{\"u}gung steht bzw. Anfangs noch gar nicht existiert, ist ein weiteres Ziel der Arbeit die Simulation und der damit einhergehenden Abstraktion der Hardware, so dass der Algorithmus auch ohne Hardware getestet werden kann.

    \item Warum ist das Ziel der Studienarbeit wichtig?\\
    Roboter werden immer mehr Teil unseres Lebens.
    Hindernis-Erkennung und ein Ausweichalgorithmus sind für die Funktion vieler dieser Roboter wichtig.
    Der, in diesem Projekt benutzte LiDAR, ist vor allem bei Hobby-Bastlern weit verbreitet.
    Daher hätte eine entsprechende Hindernis-Erkennung und dazugehöriger Ausweichalgorithmus viele Anwendungsm{\"o}glichkeiten

    \item Wie ist die Studienarbeit aufgebaut?\\
    ???

    \item Welche Probleme haben Vorgänger gehabt?\\
    - Mangelnde Hardwarestärke
    - Kein passender Algorithmus
    - Umsetzung mit ROS nicht einfach / Sinnvoll

\end{enumerate}

\section{Überblick Hardware und Software}

\subsection{Hardware}
\begin{enumerate}
    \item Raspberry PI\\
    \item LiDAR\\
\end{enumerate}

\subsection{Software}
\begin{enumerate}
    \item ROS oder alles selbst programmieren\\
\end{enumerate}

\section{SLAM-Algorithmus}

\begin{enumerate}
    \item Was ist SLAM?\\
    \item Wie funktioniert SLAM?\\
    \item Welche SLAM-Algorithmen gibt es?\\
    \item Welcher SLAM-Algorithmus ist für uns geeignet?\\

\end{enumerate}
\section{Simulation des Ausweichalgorithmus}

\begin{enumerate}
    \item Warum simulieren?\\
    \item Was simulieren?\\
    \item Woher bekommen wir die Daten?\\
    \item Wie werden die Daten verarbeitet?\\
    \item Wie werden die Daten visualisiert?\\
\end{enumerate}

\section{Implementierung des Ausweichalgorithmus}

\begin{enumerate}
    \item Zugirff auf die Daten\\
    \item Verarbeitung der Daten\\
    \item Ausweichalgorithmus\\
    \item Welche Technologien werden verwendet?\\
\end{enumerate}


\section{Fazit}
\begin{enumerate}
    \item Was haben wir erreicht?\\
    \item Etspricht das dem erhofften Ergebnis?\\
\end{enumerate}
\end{document}